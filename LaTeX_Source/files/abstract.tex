\begin{abstract}
\setcounter{page}{3}
\addcontentsline{toc}{chapter}{Abstract}
H�rten eines Systems bedeutet, die Sicherheit des Systems zu erh�hen. Erreicht wird dies durch geeignete Konfiguration der Programme und durch Abschalten der nicht ben�tigten Dienste. Im Rahmen dieser Arbeit wurden die verschiedenen Aspekte des H�rtens eines Linux Systems gezeigt. Daraus resultierte eine Liste von Vorschl�gen, die beim H�rten eines Systems beachtet werden sollten. Diese Liste umfasst folgende Ebenen eines Systems: Physisch, Lokal, Datei- und Dateisysteme, Passw�rter und Verschl�sselung, Betriebssystem-Kern und Netzwerk. Besonders ausf�hrlich wurde PAM (Pluggable Authentication Module) beschrieben, da dieses Verfahren sehr verbreitet ist und PAM viele M�glichkeiten bietet, die Sicherheit der Authentifizierung zu erh�hen. Alle Angaben, die das H�rten des Systems betreffen, wurden Distributionsunabh�ngig gemacht.

Weiter wurde eine �bung erarbeitet, die Fragen und deren L�sungen zu den Themen H�rten eines Systems, PAM, ARP (Address Resolution Protocol) und rlogin (remote login) beinhaltet. F�r diese �bung steht ein geh�rterter und ein nicht geh�rteter Red Hat Linux Server, fertig konfiguriert, auf einer CD-Rom bereit. Diese Server k�nnen direkt in eine VMware-Umgebung geladen und gestartet werden. Um die �bungsserver starten zu k�nnen braucht man eine VMware Version (auch als Trial-Version erh�ltlich) f�r das entsprechende Betriebssystem. Dadurch ist die �bungsdurchf�hrung plattformunabh�ngig.


\end{abstract}
