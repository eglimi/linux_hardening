\chapter{Einleitung}
\label{cha:Einleitung}

Diese Arbeit besch�ftigt sich mit dem H�rten eines System. Im Internet und in der Literatur stolpert man oft �ber diesen Begriff, wenn �ber Sicherheit geschrieben wird. Da der Begriff und der Vorgang des H�rtens nicht im eigentlichen Sinne definiert sind, gibt es sehr unterschiedliche Meinungen dar�ber, was H�rten bedeutet und was es alles umfasst. Die vorliegende Arbeit soll einen Eindruck vermitteln, was unter H�rten eines Systems verstanden werden kann. Sie kann keine allgemeing�ltige Referenz sein, da je nach Einsatzgebiet eines Systems der Begriff unterschiedlich aufgefasst werden kann. Die Arbeit umfasst eine Zusammenstellung von Verfahren, die nach umfassendem Studium der Literatur den Autoren als wichtig erschienen. Welche davon im einzelnen Falle angewendet werden, h�ngt von der pers�nlichen Einstellung und der Verwendung ab. Da diese Arbeit nicht den Anspruch hat, jedes Gebiet der Sicherheit zu durchleuchten, beschr�nkt sie sich auf die folgenden Punkte:
\begin{itemize}
	\item	H�rten eines Systems 
	\item	Linux als Betriebs- und Serversystem
	\item	VMWare
\end{itemize}
und l�sst folgende Punkte aus:
\begin{itemize}
	\item	Intrusion Prevention und Produkte die dieses Prinzip implementieren (z.B. Stormwatch)
	\item	Secure Application Environment (basierend auf dem Produkt PitBull LX)
	\item	Intrusion Detection
	\item	Windows als Betriebssystem
	\item	PitBull
\end{itemize}

Im Zusammenhang mit H�rten treten in der Literatur verschiedene Begriffe auf, die z.T. verwirrend sein k�nnen. Im Englischen wird von "`System Hardening"'\index{System Hardening}, "`OS Hardening"'\index{OS Hardening} oder von "`Application Hardening"'\index{Application Hardening}  gesprochen. Mit dieser Unterteilung will man darauf hinweisen, dass es verschiedene Ebenen\index{H�rten!Ebenen} des H�rtens gibt. Diese sind in Tabelle \ref{tab:hardEbenen} zusammengefasst und erkl�rt. Wenn immer diese Unterteilung nicht von Bedeutung ist, wird der deutsche Begriff \emph{H�rten} bzw. \emph{H�rten eines Systems} f�r alle Ebenen verwendet. Die Definition, was darunter zu verstehen ist, wird in Kapitel \ref{sec:WasIstHaerten} gegeben.

\LTXtable{\linewidth}{./files/inc/tables/hardEbenen}

Weitere in diesem Dokument verwendeten Begriffe und Abk�rzungen sind in einem Glossar auf Seite \pageref{cha:Glossar} zusammengefasst.

Das Literaturverzeichnis mit s�mtlichen von uns verwendeten Referenzen befindet sich auf Seite \pageref{bibliography}.

Als Resultat der Arbeit ist eine �bung vorgesehen, die im Fach Internetsicherheit die M�glichkeit gibt, die hier erkl�rten Prinzipien und Verfahren selbst auszutesten, um sie besser zu verstehen.

Die Texte wurden bewusst nicht immer in rein \glqq technischer Sprache\grqq\ verfasst. Die Autoren hoffen, dass die Texte dadurch angenehmer und verst�ndlicher zu lesen ist.



