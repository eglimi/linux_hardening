\begin{longtable}{|l|X|X|X|X|}
\caption[Installation der Server Variante 1]
{\label{tab:Serverinstallationen1} Variante 1 der Serverinstallationen f�r die �bung}\\
\hline
	\textbf{Kategorie} & \textbf{Attacke} & \textbf{Server 1} & \textbf{Server 2} & \textbf{Server 3}\\
	\hline
	\hline
		&Passw�rter cracken &shadow Passwort ausgeschaltet & shadow Passwort eingeschaltet	& \multirow{6}{35mm}{Auf diesem Server ist PitBull installiert. Sonst entspricht die Installation derjenigen von Server 1.} \\
	\cline{2-4}
	OS &ARP-Attacke 		& default Einstellungen 		& statisches ARP Cache, log Eintr�ge, arpwatch & \\
	\cline{2-4}
	&Trojanisches Pferd & - & Einschr�nkungen des Benutzers, Ausf�hrbarkeit im System einschr�nken & \\	
	\cline{1-4}
	Applikation &Attacke gegen ssh1 & ssh als Dienst verwenden und Erkl�rungen dazu machen& ssh1 updaten auf ssh2, oder sicher konfigurieren & \\
	\cline{2-4}
		& ftp (buffer overflow) & default Einstellungen & ftp Dienst sicherer konfigurieren & \\
	\cline{2-4}
	&Analysetools & Ungenutzte Dienste aktiviert (Bsp. Finger) & Ungenutzte Dienste deaktivieren &\\
	\hline
\end{longtable}
%
%\begin{longtable}{|X|X|X|X|X|}
%\caption[Installation der Server]
%{\label{tab:Serverinstallationen} Varianten der Installationen Serverinstallationen f�r die �bungen}\\
%\hline
	%Kategorie des H�rtens&\textbf{Attacke} &\textbf{Server 1} &\textbf{Server 2} &\textbf{Server 3}\\
	%\hline
	%	&Passw�rter cracken &shadow Passwort ausgeschaltet & shadow Passwort eingeschaltet	& Auf diesem Server ist \\
	%\cline{2-4}
	%OS &ARP-Attacke 		& default Einstellungen 		& statisches ARP Cache, log Eintr�ge, arpwatch & PitBull installiert.\\
%	\cline{2-4}
%	&Trojanisches Pferd & - & Einschr�nkungen des Benutzers, Ausf�hrbarkeit im System einschr�nken & \\	
%	\cline{1-4}
%	Applikation &Attacke gegen ssh1 & ssh als Dienst verwenden und Erkl�rungen dazu machen& ssh1 updaten auf ssh2, oder sicher konfigurieren & Sonst entspricht die Installation\\
%	\cline{2-4}
%		& ftp (buffer overflow) & default Einstellungen & ftp Dienst sicherer konfigurieren & derjenigen von Server\\
%	\cline{2-4}
%	&Analysetools & Ungenutzte Dienste aktiviert (Bsp.Finger) & Ungenutzte Dienste deaktivieren & 1.\\
%	\hline
%\end{longtable}
