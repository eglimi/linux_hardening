\chapter*{Aufgabenstellung}
\label{cha:Aufgabenstellung}
\addcontentsline{toc}{chapter}{Aufgabenstellung}

%----------
\section*{Short Description}
\label{sec:ShortDescription}
Nowadays any system connected to the Internet will be attacked  by attackers anywhere in the world. Systems may be compormised due to suboptimal configurations (e.g. weaknesses in applications, services and in the operation system). To decrease this risk of succesful attacks system administrators should \glqq harden\grqq\ their systems. There exist specific checklists and strategies to harden a given operating system. Argus Systems provides a so called Secure Application Environment for Linux Systems, named PitBull LX. This Software protects against application security flaws by isolating applications in separate security compartments.

The result of this thesis should lead to a lab, in which the hardening principles are outlined. The attendees of the lab should get hands on experience investigating  various systems (a standard server, a hardened server).

%----------
\section*{Tasks}
\label{sec:Tasks}
\begin{itemize}
	\item	Description of the Hardening Principles (properties of the servers, hardening concepts)
    \item	Find examples which show the differences between a standard and a hardened system
	\item	Installation and configuration of the three systems on one machine (VMWare)
    \item	Set up a lab in which students can experience the hardening principles 
\end{itemize}

\pagebreak

%----------
\section*{Technologies}
\label{sec:Technologies}
In this thesis, the students will focus on the following technologies and subjects
\begin{itemize}
	\item	Server hardening principles
    \item	Linux as server
    \item	VMWare
    \item	Labs for students
\end{itemize}

\vspace*{100mm}
\noindent
\textbf{Unterschrift} \dotfill 